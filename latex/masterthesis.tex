\documentclass[a4paper,12pt,oneside]{article}

%% Verwendete Pakete:
\usepackage[english]{babel} % für die deutsche Sprache
\usepackage{caption} % Für schönere Bildunterschriften
\usepackage[T1]{fontenc} % Schriftkodierung (Für Sonderzeichen u.a.)
\usepackage[utf8]{inputenc} % Für die direkte Eingabe von Umlauten im Editor u.a.
\usepackage{fancyhdr} % Für Kopf- und Fußzeilen
\usepackage{lscape} % Für Querformat

\usepackage{lmodern} % Schriftart "Latin Modern"
%\usepackage{garamond} % Schriftart "Garamond"

%Sans Serif-Schriften:
%\usepackage[scaled]{uarial}
%\usepackage[scaled]{helvet}
%%--------------
\usepackage[normalem]{ulem} % Für das Unterstreichen von Text z.B. mit \uline{}
\usepackage[left=3cm,right=2cm,top=1.5cm,bottom=1cm,
includeheadfoot,headsep=1cm,
footskip=1cm,headheight=14.599pt]{geometry}


\usepackage{graphicx} 
\usepackage{epstopdf}

% Pakete für Tabellen
\usepackage{tabularx} % Einfache Tabellen
\usepackage{longtable} % Tabellen als Gleitobjekte (für die Aufteilung bei langen 
 %Tabellen über mehrere Seiten)
\usepackage{multirow} % Für das Verbinden von Zeilen innerhalb einer Tabelle mit

% (Zusatz-)Pakete für Formeln
\usepackage{amsmath}
\usepackage{amsthm}
\usepackage{amsfonts}

\usepackage{setspace} % Paket zum Setzen des Zeilenabstandes
% INFO: Zeilenabstand setzen:
%
% Befehle:
% - \singlespacing  => 1-zeilig (Standard)
% - \onehalfspacing => 1,5-zeilig
% - \doublespacing  => 2-zeilig 
\onehalfspacing % Zeilenabstand auf 1,5-zeilig setzen

% Farbboxen (für die Merkkästen in dieser Vorlage):
\usepackage{tcolorbox}
\tcbset{colback=white,colframe=orange,
        fonttitle=\bfseries}

\usepackage[colorlinks,pdfpagelabels,pdfstartview=FitH,
bookmarksopen=true,bookmarksnumbered=true,linkcolor=black,
plainpages=false,hypertexnames=false,citecolor=black]{hyperref} % Für Verlinkungen

%%%%%%%%%%%%%%%%%%%%%%%%%%%%%%%%%%%%%%%%%%%%%
%% DOKUMENT                                %%
%%%%%%%%%%%%%%%%%%%%%%%%%%%%%%%%%%%%%%%%%%%%%
\begin{document}
    \pagenumbering{gobble}
    
    % Deckblatt
    \pagestyle{empty}
    \begin{titlepage}
    \begin{figure}[!ht]
        % \flushleft
            \includegraphics[width=0.26\textwidth]{./sources/logo_TH-Koeln_CMYK_22pt.png}
    \end{figure}

    \begin{center}
      \Large
      Cologne University of Applied Sciences\\
      Faculty of Computer Science and Engineering Science\\
      \hrule\par\rule{0pt}{2cm} % Horizontale Trennlinie  mit 2 cm Abtand nach unten erzeugen
      \LARGE
      \textsc{M A S T E R  T H E S I S}\\
      \vspace{1cm} % Vertikaler Abstand von 1cm erzeugen
      \huge
      Texture Asset generation through Transformer Models\\
      \vspace{1.5cm}
      \large
      Cologne University of Applied Sciences\\
      Campus Gummersbach\\
      Master Digital Sciences\\ 
      \vspace{1.0cm}
      written by:\\
      \textsc{Dennis Goßler}\\
      11140150\\
      \vspace{1.5cm}
      \begin{tabular}{ll} % Einfache Tabelle ohne Rahmen, mit 2 Spalten erzeugen
          \textbf{First examiner:} & Prof. Dr. Olaf Mersmann \\
          \textbf{Second examiner:} & Prof. Dr. Boris Naujoks \\
      \end{tabular}
    
    \end{center}    
    \end{titlepage}
    
    \newpage
    
    
    \begin{abstract}
    This thesis explores the adaptation of Transformer architectures, traditionally used in natural language processing, for generating texture assets in digital environments, such as video games. The primary focus is on developing and evaluating two novel models: the Column Image Transformer (CIT) and the Spiral Image Transformer (SIT). Both models aim to generate texture assets from seed images, photos, or drawings.

The CIT model processes images by segmenting them into vertical columns and predicts subsequent pixels within a column using the context provided by previous pixels. In contrast, the SIT model analyzes pixels in a spiral pattern to capture a broader context within the image, which can be important for generating complex textures. These models were trained on a high-performance computing system, utilizing a dataset consisting of various game textures.

Experimental results indicate that both models have a basic understanding of the colors, positions, and space of texture assets. With further work, such as using a larger dataset and scaling up the models, high-quality images could be generated and produced. Additionally, future work could explore the integration of text descriptions to guide texture generation, potentially enhancing the model's utility in real-world applications.
    \end{abstract}
    
    \newpage
    
    % Inhaltsverzeichnis
    \tableofcontents
    
    \newpage
    \pagestyle{fancy} % Kopf- und Fußzeilen aktivieren (=> Paket "fancyhdr")
    
    
    \addcontentsline{toc}{section}{List of Figures} % Manuellen Eintrag im Inhaltsverzeichnis erzeugen
    \listoffigures
    \newpage
    
    \addcontentsline{toc}{section}{Tabellenverzeichnis}
    \listoftables
    \newpage
    
  
    \pagenumbering{arabic}
    
    \section{Introduction}\label{introduction}  
    
In this thesis, we will investigate the use of a traditional Transformer model to generate texture assets to use as floor texture in e.g. in video games. Transformers are usually used to ... . But in this thesis, the focus is to use them to generate the next Pixel in a texture. Multiple different datasets of textures will be used from the internet to train the model. The final developed models will be trained on a GPU cluster in Berlin. The models will be evaluated on a set of metrics and the results will be compared to other models.

\subsection{Related work}
    

\subsection{Infrastructure for Model Development}

To develop and train the models in this thesis, a powerful computing infrastructure is necessary to manage the extensive datasets and the substantial computational requirements for model training. Unlike conventional development environments where a standard laptop or desktop may suffice, most of the models in this thesis demand a more capable infrastructure. Therefore, a high-performance computing system situated in Berlin is used for the model training processes. This system contains an array of NVIDIA Tesla A100 80 GB GPUs, Ice Lake 8360Y CPUs and a significant quantity of RAM. Such a configuration, especially the substantial GPU memory, enables the training and execution of larger models that would be possible on a home workstation. The development of these models is carried out using Python and PyTorch, with the code being crafted in Visual Studio Code and managed through version control with Git. The model development and initial code testing are done on a local machine, reserving the high-performance system exclusively for the final training phases. This approach diverges from standard practices, where often both development and execution occur on the same development platform. Ensuring the code is free of errors prior to giving the task of training the model to the high-performance computing system is crucial, as discovering bugs in the training process can be exceedingly time-consuming. For instance, to endure a training session that extends for 30 hours, only to realize it terminated prematurely due to script errors.

\subsection{Data}
    
This section describes the methods used for gathering, cleaning, and analyzing data in a research thesis on textures. Essential for training a machine learning model, the data is carefully collected from various sources, cleaned to maintain uniformity, and examined for patterns, with a focus on color distribution.


\subsubsection{Data Retrieval}
On the internet, a wide variety of textures can be found, but not all of them are suitable for this task. The textures should be seamless, devoid of shadows, and free from any objects. Textures of floors, such as carpets, tiles, wood, concrete, and more, were utilized. Two approaches were employed to acquire the data for this thesis.

\begin{itemize}
    \item Web Data Collection

    The data for this project was obtained from various online sources. Numerous free texture providers, such as textures.com, texturehaven.com, and others, were utilized for data acquisition. Due to the limitation of downloading one texture at a time from most websites, a series of scripts were developed to compile a list of suitable textures and automate the downloading process. These scripts were created using UiPath and Python.
    
    \item Video Game Textures
    
    The second approach involved using textures from video games. The advantage of this approach is that these textures are already seamless and often of high quality and quantity. However, a drawback is that these textures can be very repetitive. To obtain these textures, downward-facing recordings of the game were made, and the textures were extracted from the video. The major challenge with this approach is the need to disable shadows and all UI elements (HUD elements) in the game, which is not always possible.
\end{itemize}

\subsubsection{Data Cleaning}

    To ensure that the data is consistent and free from elements that could corrupt the model, various cleaning steps were applied. For example, all images containing 3D objects were removed, especially those gathered from video games. During the recording of the floor, unwanted debris or pieces of wood were often present, and all extracted frames were manually checked.

    In the case of web-gathered textures, there were different folder structures, and it was necessary to standardize them across all data folders. Additionally, some of them had associated files that were irrelevant to this use case and needed to be discarded.

    All the images were in high-definition (HD) quality, with a height of approximately 1024 pixels.


    \begin{table}[h]
        \centering
        \begin{tabular}{|c|c|c|}
            \hline
            Dataset & Size & Number of Images \\
            \hline
            FreePBR & 452.0 MB & 263 \\
            Polyhaven & 298.0 MB & 439 \\
            Poliigon & 70.4 MB & 49 \\
            Minecraft-Textures &  636.0 MB & 493 \\
            CsGoFloor-Textures & 18.3 GB & 44540 \\
            \hline
            Combined & 20.2 GB & 45784 \\
            \hline
        \end{tabular}
        \caption{Datasets collected for this thesis}
        \label{tab:datasets}
    \end{table}

    \subsubsection{Patterns in the data}
    
    To examine whether the dataset encompasses a broad spectrum of colors, multiple plots are created. These plots illustrate the color distribution within the datasets, providing insights into the diversity of colors present. Prior to plotting, a comprehensive pixel count across all images is conducted. For instance, if an image features 10 pixels of the color $(255, 0, 0)$, this count is added to a dictionary. Should the subsequent image in the dataset contain 5 pixels of the same color, these are also incorporated into the dictionary, cumulating a total of 15 for that specific color. This process is repeated for each color encountered, aggregating the counts to yield the overall color frequency within the dataset.

\begin{lstlisting}[language=Python]
  color_counts = {} 
  for i, (data, _) in enumerate(dataset):
      # data is a tensor of shape (3, height, width) 
      pixel_rgb_array = (data.view(3, -1).t() * 255).to(torch.int32)
      
      for pixel_color in map(tuple, pixel_rgb_array):
          if color in color_counts:
              color_counts[pixel_color] += 1
          else:
              color_counts[pixel_color] = 1
\end{lstlisting}

    After analyzing the dataset through this method, visual representations of the color distributions were produced using Python and Matplotlib. These plots provide a three-dimensional view of the RGB color space, where the X, Y, and Z axes correspond to the Red, Green, and Blue color values, respectively, each ranging from 0 to 255. 

    \[
    \text{size} = \log(\text{count of color}) \times 20
    \]

    The size of each plotted point is calculated based on the logarithm of the color count, scaled by a factor of 20.
    
    

    \begin{figure}[htbp]
        \centering
        
        \begin{subfigure}{.33\textwidth}
          \centering
          \includegraphics[width=\linewidth]{../code/dataAnalysis/output/FreePBR.png}
          \caption{FreePBR}
          \label{fig:dataset-FreePBR}
        \end{subfigure}%
        \hfill
        \begin{subfigure}{.33\textwidth}
          \centering
          \includegraphics[width=\linewidth]{../code/dataAnalysis/output/Polyhaven.png}
          \caption{Polyhaven}
          \label{fig:dataset-Polyhaven}
        \end{subfigure}%
        \hfill
        \begin{subfigure}{.33\textwidth}
          \centering
          \includegraphics[width=\linewidth]{../code/dataAnalysis/output/Poliigon.png}
          \caption{Poliigon}
          \label{fig:dataset-Poliigon}
        \end{subfigure}
        
        \vspace{1cm} % Vertikaler Abstand zwischen den Reihen
        
        \begin{subfigure}{.33\textwidth}
          \centering
          \includegraphics[width=\linewidth]{../code/dataAnalysis/output/Minecraft_1024x.png}
          \caption{Minecraft-Textures}
          \label{fig:dataset-Minecraft-Textures}
        \end{subfigure}%
        \hfill
        \begin{subfigure}{.33\textwidth}
          \centering
          \includegraphics[width=\linewidth]{../code/dataAnalysis/output/CsGoFloor_1080x.png}
          \caption{CsGoFloor-Textures}
          \label{fig:dataset-CsGoFloor-Textures}
        \end{subfigure}%
        \hfill
        \begin{subfigure}{.33\textwidth}
            \centering
            \includegraphics[width=\linewidth]{../code/dataAnalysis/output/combined.png}
            \caption{Combined}
            \label{fig:dataset-Combined}
        \end{subfigure}%
        \hfill
    \end{figure}

    In the figure above, the color distributions of the individual datasets are shown. The first five subfigures represent the color distributions of the individual datasets, while the last subfigure (\ref{fig:dataset-Combined}) shows the combined color distribution of all datasets. The color distributions of the individual datasets are quite similar, except for the Minecraft-Textures dataset, which is way more colorful than the others. The combined figure is a combination of all the individual datasets, and it is evident that the color distribution is quite diverse. This is a positive sign, as it indicates that the dataset is not focused on only a specific color spectrum.

    \subsubsection{Data Synchronization}

    In the thesis, a manual data synchronization routine is established to maintain data consistency between the supercomputer located in Berlin and the local workstation.

\subsection{Training process}
    (gpu cluster göthingen, my GPU, ...)

\subsection{Models}
    (LLMs, basic idea, roll model, spiral model)

    \subsubsection{Column Image Transformer}

    In the context of this thesis, a model termed the Column Image Transformer (CIT) has been conceptualized and developed. This model embodies an adaptation of the conventional transformer architecture. Distinctively, the CIT model diverges from traditional image processing techniques by segmenting the image into vertical slices or columns of pixels. This segmentation allows for a method where each column is processed on its own, following the "B" batch dimension in the model's structure.

    The model assigns position embeddings to every pixel in a column, indicated by the "H" dimension. This approach enables the model to predict the properties of subsequent pixels within the same column, enhancing the model's ability to reconstruct image content and have a basic understanding of the context. Each pixel includes three values corresponding to the Red, Green, and Blue (RGB) color channels. These channels are processed across the model's layers as the "C" dimension.

    \subsubsection{Spiral Image Transformer}

    The second approach is represented by the Spiral Image Transformer (SIT). Unlike its predecessor, the Column Image Transformer (CIT), the SIT model employs a contextually spiral pattern. This architecture enables the generation of images starting from a central point and expanding outward (see ...). In the SIT model, the batch dimensions correspond to distinct images, whereas the H dimension represents the spiral context. Similar to the CIT model, the C dimension denotes the color channels.

    One of the pivotal enhancements of the SIT model is its ability to analyze adjacent pixels on the horizontal axis, in contrast to the CIT model's limitation to columnar pixel analysis. This feature is particularly beneficial for interpreting textures with intricate patterns, such as diagonal ones, thereby offering an advantage over the Column Image Transformer. However, it is important to note that the SIT model operates within a constrained area of the image due to its 2D context. This limitation necessitates the use of only a portion of the image area, specifically a sector determined by the square root of the total area available to the Column Image Transformer (CIT), with an equal context length.
    
 
    \newpage

    \section{Experiment}
    
    \subsection{The Foundation of the Models}

    To build the models, a set of Python libraries is utilized, mainly PyTorch, NumPy, TensorBoard, and Einops. PyTorch serves as the core tool for constructing and training the models.
    
    The models discussed in this thesis are based on a Python script by A. Karpathy named NanoGPT \autocite{nanoGPTkarpathy2023}. This script demonstrates a straightforward and accessible GPT implementation. 
    
    \subsubsection{Data Handling}
    
    For data management, a script named dataSetCombiner is developed to load and return a combined dataset from specified image folders with optional transformations. The function combines the 33 different sources from multiple directories into a single dataset, also offering the option to apply various image transformations such as cropping, flipping, gray scaling, and color jitter.
    
    It allows for the selection between 512x512 and 1024x1024 pixel datasets, depending on the use case. Parameters can be adjusted to tailor the dataset to specific needs, such as image size, the particular dataset to load, and the option to include multiple instances of the dataset. Furthermore, it can randomly flip images vertically or horizontally to augment the dataset, thereby preventing overfitting and improving model generalization. It also supports color jittering, allowing for random adjustments in image brightness, contrast, saturation, and hue, which introduces further variability into the dataset when needed. Additionally, an option to convert images to grayscale is available.
    

\begin{lstlisting}[language=Python]
    def getDataSet(path, dataset_name, size_x, size_y, repeatData=1, random_vertical_flip=False, random_horizontal_flip=False, crop_type='random', grayscale=False, color_jitter=False, jitter_brightness=0, jitter_contrast=0, jitter_saturation=0, jitter_hue=0):
\end{lstlisting}

    \subsubsection{Monitoring Training Progress}
    
    The training process is monitored using TensorBoard, a tool that helps visualize different aspects of training. In this thesis, TensorBoard is particularly useful for tracking training and validation loss through plots. It also allows for the viewing of the input images and the outputs generated by the models.    
    

\subsection{Roll model}
    (explanation, generating new content, , , )    
    
    \subsubsection{Classification or Regression}
    \subsubsection{Discriminator}


\subsection{Spiral model}
    (explanation, Data to Spiral form, positional embedding, )

    \subsubsection{Data to Spiral form}



    \subsubsection{Positional embedding}

\subsection{Problems}
    (layer norm(sigmoid vs clamp), color shift to gray (illustrations of average color), Text tokens vs imgs tokens)
    \newpage
    
    \section{Conclusion}\label{conclusion}  
    
This section focuses on the evaluation of the performance of the CIT and the SIT.

\subsection{Evaluation of the models}
    
    As a general conclusion it can be said that the data hints that it is possible to generate quickly new assets trow this approach. Due to the fact that the models are relatively small and the training dataset is not that big the output of the models is 

    \subsubsection{performance}
    One the big question is if it is possible to generated assets for video game. Here we can split it into two parts. The first part is to generate assets beforehand in the development cylce and in the second part is is the model so efficient that it is possible to generate assets in real time on a local machine.

    \subsubsection{quality}


\subsection{Execution locally and on the cloud}
    
    

\subsection{Further research}

    \subsubsection{Discriminator}
    %TODO: Add more information about the discriminator
    To get a better result a discriminator can be used to enhance the output of the model. In this case, a discriminator is added after the model prediction.

    For these XX new rows are generated and then the discriminator checks if the generated content is artifice or an original image. In a perfect scenario the discriminator can't distinguish between the generated content and the original content and the loss will balance out at 50

    \subsubsection{LLM Scaling Laws}


    \subsection{Stable diffusion/ GANs with convolutional neural network} 
    
    
    % Anhang
    \newpage
    \setcounter{section}{0} % Nummerierung der Gliederungsebene "section" auf 0 setzen
    \renewcommand*\thesection{\Alph{section}} % Nummerierungsart für die Gliederungsebene "section" 
    % auf Großbuchstaben setzen
    \section{Appendix}\label{appendix}
    % syntax: none
\subsection{Dataset Plots in the Lab Color Space}\label{sec:app_labPlots}

\begin{figure}[H]
    \centering
    \foreach \i in {-1,...,18}{
      \begin{subfigure}[t]{0.23\textwidth}
        \centering       
        \ifnum \i=-1
            \includegraphics[width=\textwidth]{../code/dataAnalysis/plots/lab/DataCombined_lab.png}
            \caption{All Data}
        \else
            \includegraphics[width=\textwidth]{../code/dataAnalysis/plots/lab/labPlot_\i}
            \caption{source \i}
        \fi

        \label{fig:lab_sub\i}
      \end{subfigure}
      % Use the modulo operation to determine if a line break should be added
      \pgfmathparse{int(mod(\i+2,4))}
      \ifnum\pgfmathresult=0
          \newline
      \else
          \hfill
      \fi
    }
    \begin{subfigure}[t]{0.23\textwidth}
        \centering
        % No image to include, so this is left empty
        \caption*{} % Empty caption
    \end{subfigure}
    \caption*{}
    \addtocounter{figure}{-1}
\end{figure}

\begin{figure}[H]
    \centering
    \foreach \i in {19,...,33}{
      \begin{subfigure}[t]{0.23\textwidth}
        \centering       
        \ifnum \i=-1
            \includegraphics[width=\textwidth]{../code/dataAnalysis/plots/lab/DataCombined_lab.png}
            \caption{All Data}
        \else
            \includegraphics[width=\textwidth]{../code/dataAnalysis/plots/lab/labPlot_\i}
            \caption{source \i}
        \fi

        \label{fig:lab_sub\i}
      \end{subfigure}
      % Use the modulo operation to determine if a line break should be added
      \pgfmathparse{int(mod(\i+2,4))}
      \ifnum\pgfmathresult=0
          \newline
      \else
          \hfill
      \fi
    }
    \begin{subfigure}[t]{0.23\textwidth}
        \centering
        % No image to include, so this is left empty
        \caption*{} % Empty caption
    \end{subfigure}
    \caption{Visualization of datasets plotted in the LAB color space}
    \label{fig:lab_all2}

\end{figure}


\newpage

\subsection{Dataset Plots in the RGB Color Space}

\begin{figure}[H]
    \centering
    \foreach \i in {-1,...,18}{
      \begin{subfigure}[t]{0.23\textwidth}
        \centering

        \ifnum \i=-1
            \includegraphics[width=\textwidth]{../code/dataAnalysis/plots/rgb/DataCombined_rgb.png}
        \caption{All Data}
        \else
            \includegraphics[width=\textwidth]{../code/dataAnalysis/plots/rgb/rgbPlot_\i}
            \caption{source \i}
        \fi
        \label{fig:rgb_sub\i}
      \end{subfigure}
      % Use the modulo operation to determine if a line break should be added
      \pgfmathparse{int(mod(\i+2,4))}
      \ifnum\pgfmathresult=0
          \newline
      \else
          \hfill
      \fi
    }
    \begin{subfigure}[t]{0.23\textwidth}
        \centering
        % No image to include, so this is left empty
        \caption*{} % Empty caption
    \end{subfigure}
    \caption*{}
    \addtocounter{figure}{-1}

\end{figure}

\begin{figure}[H]
    \centering
    \foreach \i in {19,...,33}{
      \begin{subfigure}[t]{0.23\textwidth}
        \centering

        \ifnum \i=-1
            \includegraphics[width=\textwidth]{../code/dataAnalysis/plots/rgb/DataCombined_rgb.png}
        \caption{All Data}
        \else
            \includegraphics[width=\textwidth]{../code/dataAnalysis/plots/rgb/rgbPlot_\i}
            \caption{source \i}
        \fi
        \label{fig:rgb_sub\i}
      \end{subfigure}
      % Use the modulo operation to determine if a line break should be added
      \pgfmathparse{int(mod(\i+2,4))}
      \ifnum\pgfmathresult=0
          \newline
      \else
          \hfill
      \fi
    }
    \begin{subfigure}[t]{0.23\textwidth}
        \centering
        % No image to include, so this is left empty
        \caption*{} % Empty caption
    \end{subfigure}
    \caption{Visualization of datasets plotted in the RGB color space}
    \label{fig:rgb_all}

\end{figure}

\newpage
\begin{landscape}
\subsection{Trained Models with hyperparameters}
\label{sec:trained_models_hyperparameters}


\begin{table}[H]
    \centering
    \begin{threeparttable}
    \caption{Model Configuration Overview}
    \scriptsize
    \begin{tabular}{|c|c|c|c|c|c|c|c|c|c|c|c|c|}
    \toprule
    MODEL & Nr. & VER. & BATCH\_SIZE & BLOCK\_SIZE & N\_EMBD & N\_HEAD & N\_LAYER & PARAMETER\tnote{*} & GPUS & DATASET & VAL\_LOSS\tnote{**}  \\
    \midrule
    \multirow{5}{*}{\rotatebox{90}{CIT}}
    & 0 & 2.1.7.0 & 128 & 128 & 128 & 6 & 6 & 1.22 m & 1 & old Dataset & 0.0082  \\
    & 1 & 2.1.8.0 & 64 & 64 & 128 & 6 & 6 & 1.21 m & 1 & x512 Dataset & 0.0035 \\
    & 2 & 2.1.8.0 Classify & 64 & 64 & 128 & 6 & 6 & 1.25 m & 1 & x512 Dataset & 1.841 \\
    & 3 & 2.1.8.0 & 64 & 256 & 512 & 8 & 8 & 25.66 m & 4 & x512 Dataset & 0.0025 \\
    & 4 & 2.1.8.0 & 32 & 511 & 640 & 9 & 9 & 45.09 m & 4 & x512 Dataset & 0.0024 \\
    \midrule
    \multirow{2}{*}{\rotatebox{90}{SIT}}
    & 5 & 5.0.1.0 & 8 & 4,095 & 128 & 6 & 6 & 1.73 m & 1 & x512 Dataset & 0.0074 \\
    & 6 & 5.0.1.0 & 4 & 4,095 & 512 & 8 & 8 & 27.62 m & 1 & x512 Dataset & 0.0027 \\
    \bottomrule
    \end{tabular}
    \begin{tablenotes}
    \item[*] Figures in millions
    \item[**] Average of the last 5 validation loss values
    \end{tablenotes}
    \end{threeparttable}
\end{table}

\end{landscape}

\newpage

\subsection{Local workstation setup}

\begin{table}[H]
    \centering
    \begin{tabularx}{0.90\textwidth}{|X|X|}
    \hline
    \textbf{Component} & \textbf{Specification} \\ \hline
    Operating System & Microsoft Windows 11 Pro \\ \hline
    CPU & AMD Ryzen 9 7900X 12 cores, 4.7 GHz \\ \hline
    GPU & NVIDIA RTX 3090 \\ \hline
    RAM & 64 GB (DDR5-6000) \\ \hline
    SSD & 2TB Samsung 990 PRO M.2 PCIe 4.0 \\ \hline
    \end{tabularx}
    \caption{Local Workstation Setup}
    \label{table:workstation_setup}
\end{table}
    
\begin{table}[H]
    \centering
    \begin{tabularx}{0.90\textwidth}{|X|X|}
    \hline
    \textbf{Component} & \textbf{Specification} \\ \hline
    CPU & 2x Intel Xeon "Ice Lake" Platinum 8360Y (36 cores per socket, 2.4 GHz) \\ \hline
    GPU & 4x Nvidia A100 (80GB HBM2, SXM) \\ \hline
    RAM & 1 TB RAM (DDR4-3200) \\ \hline
    SSD & 7.68 TB NVMe local SSD \\ \hline
    \end{tabularx}
    \caption{Single compute node}
    \label{table:server_setup}
\end{table}

\newpage

\subsection{Transformer layer implementation}
\label{sec:transformer_layer_Python}

\begin{lstlisting}[language=Python]
class Head(nn.Module):
""" one head of self-attention """

    def __init__(self, head_size):
        super().__init__()
        self.key = nn.Linear(N_EMBD, head_size, bias=False)
        self.query = nn.Linear(N_EMBD, head_size, bias=False)
        self.value = nn.Linear(N_EMBD, head_size, bias=False)
        self.register_buffer('tril', torch.tril(torch.ones(BLOCK_SIZE, BLOCK_SIZE)))

        self.dropout = nn.Dropout(DROPOUT)

    def forward(self, x):
        # input of size (batch, time-step, channels)
        # output of size (batch, time-step, head size)
        B,T,C = x.shape
        k = self.key(x)   # (B,T,hs)
        q = self.query(x) # (B,T,hs)
        # compute attention scores ("affinities")
        wei = q @ k.transpose(-2,-1) * k.shape[-1]**-0.5 # (B, T, hs) @ (B, hs, T) -> (B, T, T)
        wei = wei.masked_fill(self.tril[:T, :T] == 0, float('-inf')) # (B, T, T)
        wei = F.softmax(wei, dim=-1) # (B, T, T)
        wei = self.dropout(wei)
        # perform the weighted aggregation of the values
        v = self.value(x) # (B,T,hs)
        out = wei @ v # (B, T, T) @ (B, T, hs) -> (B, T, hs)
        return out

class MultiHeadAttention(nn.Module):
""" multiple heads of self-attention in parallel """

    def __init__(self, num_heads, head_size):
        super().__init__()
        self.heads = nn.ModuleList([Head(head_size) for _ in range(num_heads)])
        self.proj = nn.Linear(head_size * num_heads, N_EMBD)
        self.dropout = nn.Dropout(DROPOUT)

    def forward(self, x):
        out = torch.cat([h(x) for h in self.heads], dim=-1)
        out = self.dropout(self.proj(out))
        return out

class FeedFoward(nn.Module):
""" a simple linear layer followed by a non-linearity """

    def __init__(self, n_embd):
        super().__init__()
        self.net = nn.Sequential(
            nn.Linear(n_embd, 4 * n_embd),
            nn.ReLU(),
            nn.Linear(4 * n_embd, n_embd),
            nn.Dropout(DROPOUT),
        )

    def forward(self, x):
        return self.net(x)

class Block(nn.Module):
""" Transformer block: communication followed by computation """

    def __init__(self, n_embd, n_head):
        # n_embd: embedding dimension, n_head: the number of heads we'd like
        super().__init__()
        head_size = n_embd // n_head
        self.sa = MultiHeadAttention(n_head, head_size)
        self.ffwd = FeedFoward(n_embd)
        self.ln1 = nn.LayerNorm(n_embd)
        self.ln2 = nn.LayerNorm(n_embd)

    def forward(self, x):
        x = x + self.sa(self.ln1(x))
        x = x + self.ffwd(self.ln2(x))
        return x
\end{lstlisting}

This code snippet shows the implementation of the transformer layer in the Column Image Transformer and the Spiral Image Transformer. The code above is heavily inspired by the implementation of the script by A.
Karpathy named NanoGPT \autocite{nanoGPTkarpathy2023} \autocite{nanogpt-lecturekarpathy2023}
    \pagenumbering{roman}
        ABC
    \newpage
        ABC
    \newpage
        ABC
    \newpage
    
    
    \pagestyle{empty}
    \pagenumbering{gobble}
    \section{Eidesstattliche Erklärung}
    
    Ich versichere, die von mir vorgelegte Arbeit selbständig verfasst zu haben.
    Alle Stellen, die wörtlich oder sinngemäß aus veröffentlichten oder nicht veröffentlichten Arbeiten anderer entnommen sind,
    habe ich als entnommen kenntlich gemacht.\\ 
    Sämtliche Quellen und Hilfsmittel, die ich für die Arbeit benutzt habe, sind
    angegeben. Die Arbeit hat mit gleichem Inhalt bzw. in wesentlichen Teilen noch keiner anderen Prüfungsbehörde vorgelegen.\\\\
    \begin{tabular}{cp{7cm}}
    & \\ 
    & \\ \hline
    \small (Ort, Datum, Unterschrift) & \normalsize \\
    \end{tabular}
    
    
    \newpage
    % Unbeschriftetes Abschlussblatt (Leere Seite)
    \thispagestyle{empty}
    
\end{document}

