
    \subsection{The Foundation of the Models}

    To build the models, a set of Python libraries is utilized, mainly PyTorch, NumPy, TensorBoard, and Einops. PyTorch serves as the core tool for constructing and training the models.
    
    The models discussed in this thesis are based on a Python script by A. Karpathy named NanoGPT \autocite{nanoGPTkarpathy2023}. This script demonstrates a straightforward and accessible GPT implementation. 
    
    \subsubsection{Data Handling}
    
    For data management, a script named dataSetCombiner is developed to load and return a combined dataset from specified image folders with optional transformations. The function combines the 33 different sources from multiple directories into a single dataset, also offering the option to apply various image transformations such as cropping, flipping, gray scaling, and color jitter.
    
    It allows for the selection between 512x512 and 1024x1024 pixel datasets, depending on the use case. Parameters can be adjusted to tailor the dataset to specific needs, such as image size, the particular dataset to load, and the option to include multiple instances of the dataset. Furthermore, it can randomly flip images vertically or horizontally to augment the dataset, thereby preventing overfitting and improving model generalization. It also supports color jittering, allowing for random adjustments in image brightness, contrast, saturation, and hue, which introduces further variability into the dataset when needed. Additionally, an option to convert images to grayscale is available.
    

\begin{lstlisting}[language=Python]
    def getDataSet(path, dataset_name, size_x, size_y, repeatData=1, random_vertical_flip=False, random_horizontal_flip=False, crop_type='random', grayscale=False, color_jitter=False, jitter_brightness=0, jitter_contrast=0, jitter_saturation=0, jitter_hue=0):
\end{lstlisting}

    \subsubsection{Monitoring Training Progress}
    
    The training process is monitored using TensorBoard, a tool that helps visualize different aspects of training. In this thesis, TensorBoard is particularly useful for tracking training and validation loss through plots. It also allows for the viewing of the input images and the outputs generated by the models.    
    

\subsection{Roll model}
    (explanation, generating new content, , , )    
    
    \subsubsection{Classification or Regression}
    \subsubsection{Discriminator}


\subsection{Spiral model}
    (explanation, Data to Spiral form, positional embedding, )

    \subsubsection{Data to Spiral form}



    \subsubsection{Positional embedding}

\subsection{Problems}
    (layer norm(sigmoid vs clamp), color shift to gray (illustrations of average color), Text tokens vs imgs tokens)