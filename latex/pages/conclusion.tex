
This section focuses on the evaluation of the performance of the CIT and the SIT.

\subsection{Evaluation of the models}
    
    As a general conclusion it can be said that the data hints that it is possible to generate quickly new assets trow this approach. Due to the fact that the models are relatively small and the training dataset is not that big the output of the models is 

    \subsubsection{performance}
    One the big question is if it is possible to generated assets for video game. Here we can split it into two parts. The first part is to generate assets beforehand in the development cylce and in the second part is is the model so efficient that it is possible to generate assets in real time on a local machine.

    \subsubsection{quality}


\subsection{Execution locally and on the cloud}
    
    

\subsection{Further research}

    \subsubsection{Discriminator}
    %TODO: Add more information about the discriminator
    To get a better result a discriminator can be used to enhance the output of the model. In this case, a discriminator is added after the model prediction.

    For these XX new rows are generated and then the discriminator checks if the generated content is artifice or an original image. In a perfect scenario the discriminator can't distinguish between the generated content and the original content and the loss will balance out at 50

    \subsubsection{LLM Scaling Laws}


    \subsection{Stable diffusion/ GANs with convolutional neural network}